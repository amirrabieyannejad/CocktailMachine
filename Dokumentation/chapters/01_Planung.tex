%!TEX root = ../main.tex

\newpage

\chapter{Planung} %  (fold)
\label{cha:planung}

\section{Kurzbeschreibung}
\label{sec:kurzbeschreibung}
Die Funktion der App ist die Steuerung einer Cocktailmaschine.

\subsection{Cocktailmaschine} 
Die Cocktailmaschine ist ein Gerät, welches über k Pumpen verfügt. Diese Pumpen werden von einem Micro-Chip gesteuert. Jede Pumpe führt von einem Flüssigkeitsbehälter zu einer kaffeemaschinenähnlichen Ausgabe, unter der man ein Glas stellen kann. Der Micro-Chip ist mit Bluetooth steuerbar.

\subsection{Funktion der App} 
Die Funktion der App ist die Steuerung einer Cocktailmaschine. Dabei sollen angezeigt werden:
\begin{itemize}
\item	die Flüssigkeiten, die an der Cocktailmaschine angeschlossen sind (Wodka, Rum, Cola, etc. ),
\item	die damit erstellbaren Cocktails (Mochito, Long Island Ice Tea, etc.),
\item	Angaben zu zusätzlichen Zutaten, die nicht von Cocktailmaschine steuerbar sind (Eis-Würfel, Zitronenscheiben, etc.)
\end{itemize}
Die Funktionen der App gehen über die reine Darstellung hinaus:
\begin{itemize}
\item	die Neuerstellung von Cocktails mit den verfügbaren Zutaten
\item	einen Auftrag aufgeben den ausgesuchten Cocktail zu mixen
Zitronenscheiben, etc.)
\end{itemize}
Ein Administrator sollte zudem:
\begin{itemize}
\item	die Flüssigkeiten einer Pumpe ändern können (Nachfüllen, andere Zutat)
\item	Zugriff auf alle auch nicht verfügbaren Zutaten und Rezepte haben
\item	das Hinzufügen neuer Zutaten
Zitronenscheiben, etc.)
\end{itemize}




\section{Entwicklungsprozess}
\label{sec:entwicklungsprozess}



\section{Team}
\label{sec:team}



\section{Risikomanagement}
\label{sec:risikomanagement}


\section{Zeitplan}
\label{sec:zeitplan}

